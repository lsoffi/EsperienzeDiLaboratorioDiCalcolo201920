\documentclass[11pt]{article}
\usepackage{graphicx,wasysym, mdframed,xcolor,gensymb,verbatim}
\usepackage{color}
\usepackage{floatflt}
\usepackage[italian]{babel}

\newcommand{\voto}[1]{[\textbf{#1} punti]}
\def\cmu{\mbox{cm$^{-1}$}}
\def\half{\frac{1}{2}}

\voffset -2cm
\hoffset -2.5cm
%\marginparwidth 0cm
\textheight 22cm
\textwidth 17cm
%\oddsidemargin  0.2cm                                                                                         
%\evensidemargin 0.4cm                                                                                         
\parindent 0pt

\begin{document}
\pagestyle{empty}

\begin{center}
{\large \bf  Laboratorio di Calcolo per Fisici, Informazioni Utili\\[2mm]}
\end{center}

{\bf User-ID:} \texttt{lcblxx}, dove $xx$ \`e il numero del gruppo a cui siete stati assegnati.
\\
{\bf Pagina web del corso (esercitazioni):}
  ({\em https://lboeri.wordpress.com/teaching/labcalc/ex/})
\\
{\bf Pagina web con le home directory}:{\em  http://labcalc.fisica.uniroma1.it/studenti/}.
\vspace{2mm}
\hrule
\vspace{2mm}
\textbf{$\RHD$ Comandi UNIX di base}
\begin{itemize}
\item Per creare una cartella: \texttt{mkdir nomecartella}.
\item Per entrare in una cartella: \texttt{cd nomecartella}.
\item Per uscire da una cartella: \texttt{cd ..}. Per tornare nella vostra
  {\em home directory} (cartella home) \`e sufficiente digitare \texttt{cd}.
\item Per vedere il contenuto di una cartella: \texttt{ls}.
\item Per copiare il contenuto di un file ({\em file1}) in un secondo file
  ({\em file2}): \texttt{cp file1 file2}.
\item Per rinominare un file da {\em old} a {\em new}: \texttt{mv old new}.
\item Per spostare o copiare un file in una cartella diversa si usano i comandi \texttt{mv} e \texttt{cp}, es: \texttt{mv file1 cartella} o  \texttt{cp file1 cartella}. Per copiare/spostare {\em tutti} i file \texttt{cp *}/ \texttt{mv *}.
\item Per cancellare (rimuovere) un file: \texttt{rm file}.
\item Per rimuovere una cartella (vuota):  \texttt{rmdir nomecartella}.
 \item Per avere informazioni su di un comando specifico si usa il comando
  \texttt{man}, es:  \texttt{man ls} fornisce tutte le indicazioni sul funzionamento del comando \texttt{ls} e le opzioni possibili.
\end{itemize}
\hrule
\vspace{2mm}
\textbf{$\RHD$ Altri Comandi utili}
\begin{itemize}
\item Per aprire (lanciare) l'editor di testo: \texttt{emacs nomefile}.
\item Per aprire (lanciare) l'editor di testo in modalit\`a {\em background}
  \texttt{emacs nomefile \&}.
\item Per {\em compilare} un programma scritto in C:
  \texttt{gcc programma.c -o programma.x} 
\item Per {\em eseguire} un programma compilato:
  \texttt{./programma.x}
\item Per {\em eseguire} un programma compilato in modalit\`a {\em background} (utile in seguito):
  \texttt{./programma.x \&}
  \item Per vedere quali programmi sono in esecuzione (utile in seguito): \texttt{top}.
\item Per eseguire un programma compilato e ridirigere l'output su un file esterno {\em output} (utile in seguito) \texttt{./programma.x > output} 
\item Per lanciare il programma {\em gnuplot} (programma per fare i grafici):
  \texttt{gnuplot}. Per uscire digitare \texttt{exit}.
\item Per accedere all'help generale di {\em gnuplot} digitare  \texttt{help}.
  Per avere informazioni su un comando specifico digitare \texttt{help nomecomando}.
\end{itemize}




\end{document}
